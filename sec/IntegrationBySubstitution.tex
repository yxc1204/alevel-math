\documentclass[lang=cn,newtx,10pt,scheme=chinese]{elegantbook}

\begin{document}
\chapter{Integration by substitution}
\section{第一类换元积分法}

\subsection{不定积分的第一类换元积分法}
对于$\int f(x) dx$,若满足以下两点,则可以用第一类换元积分法计算。
\begin{itemize}
\item $f(x)$形如$\tilde{f}(g(x))g'(x)$,即$\int f(x)dx = \int \tilde{f}(g(x))g'(x)dx = \int \tilde{f}(g(x))dg(x)$;
\item 函数$\tilde{f}(u)$的不定积分容易计算,即$\int \tilde{f}(u)du = \tilde{F}(u)$易知.
\end{itemize}
\begin{theorem}[第一类换元积分法]
对于不定积分$\int f(x)dx$,用$u=g(x)$对原式作变量代换,于是:
  \begin{equation*}
    \int f(x)dx = \int \tilde{f}(g(x))g'(x)dx = \int \tilde{f}(g(x))dg(x)
    \overset{\rm{*}}= \int \tilde{f}(u)du = \tilde{F}(u) + C \overset{\rm{*}}= \tilde{F}(g(x))+C
  \end{equation*}
\end{theorem}

对不定积分使用第一类换元积分法时,需要注意以下几点:
\begin{itemize}
\item 上式第一个*表示令$g(x)=u$对原式作变量代换,第二个*表示回代$u=g(x)$得到不定积分的结果
  \item 不能遗漏第二个*的回代,因为$\int f(x)dx$的计算结果应该是关于$x$的函数
\end{itemize}
%例题讲解
\begin{example}
  计算 $\int \cos x \sin^2 x dx$.
\end{example}
\begin{solution}

\end{solution}

% 基础练习-给出换元法
基础练习:给出换元方式$u=g(x)$,计算不定积分 
例子一$\int \frac{x}{\sqrt{x^2+4}}dx$. Use the substitution $u=x^2+4$;
例子二$\int x^2(2x^3+5)^4dx$. Use the substitution $u=x^2+4$;

% 进阶练习-不给换元法
进阶练习:自己寻找换元方式$u=g(x)$,计算不定积分
例子一$\int x \sqrt{x^2+3}dx$;
例子二$\int \tan x dx$;
例子三$\int \sin^3x\cos^4xdx$

\subsection{定积分的第一类换元积分法}
类似地,对于$\int_b^af(x)dx$,若对应不定积分有如上形式,则可以用定积分中的第一类换元积分法计算。
\begin{theorem}[第一类换元积分法]
  类似地,对于定积分$\int_{b}^{a} f(x)dx$,用$u=g(x)$对原式作变量代换,有:
  \begin{equation*}
    \int_b^a f(x)dx = \int_b^a \tilde{f}(g(x))g'(x)dx = \int_b^a \tilde{f}(g(x))dg(x)
    \overset{\rm{*}}= \int_{g(b)}^{g(a)} \tilde{f}(u)du = \tilde{F}(g(a))-\tilde{F}(g(b)) 
  \end{equation*}
\end{theorem}

对定积分使用第一类换元积分法时,需要注意以下几点:
\begin{itemize}
\item 换元后的定积分$\int_{g(b)}^{g(a)}\tilde{f}(u)du$中的上下限$g(a)$和$g(b)$必须与原来定积分的上下限$a$和$b$相对应,无须考虑$a$与$b$之间大小问题。
  \item 与不定积分的第一类换元积分法相比,定积分只需要求值,无须算出具体原函数,故最后无须回代$u=g(x)$
\end{itemize}
例子一$\int_2^{5} x \sqrt{x^2+3}dx$;
例子二$\int_{0}^{\frac{\pi}{2}} \tan x dx$;
例子三$\int_0^{\frac{\pi}{2}} \sin^3x\cos^4xdx$

\section{第二类换元积分法}
\subsection{不定积分的第二类换元积分法}
对于无法直接求出,也不适用第一类换元积分法的不定积分$\int f(x)dx$。
第二类变换指的是用$x=\varphi(t)$代入
若满足以下条件,则可以用第二类换元积分法计算。
\begin{itemize}
\item 能找到合适换元$x=\varphi(t)$, 变换后不定积分形式更简单。
变换规则:$\int f(x)dx = \int f(\varphi(t))d\varphi(t) = \int f(\varphi(t))\varphi '(t)dt$ 
\item 函数$f(\varphi(t))\varphi '(t)$的不定积分容易计算,即$\int f(\varphi(t))\varphi '(t)dt = \tilde{F}(u)$易知.
\end{itemize}

\subsection{定积分的第二类换元积分法}
\end{document}